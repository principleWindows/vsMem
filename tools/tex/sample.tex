% A size substituation of 0.5pt is harmless, but if you want to avoid it load the fix-cm package:
\RequirePackage{fix-cm}
%\documentclass{book}
\documentclass[a4paper,12pt]{ctexbook}	%A4纸,字体10pt,包封的,ctex书籍排版
%\let\Tiny=\tiny
\usepackage[UTF8]{ctex}
\usepackage{amsmath}
\usepackage{lmodern}
% 下面一行使用 pdf 书签
\usepackage[breaklinks,colorlinks,linkcolor=black,citecolor=black,urlcolor=black]{hyperref}
%\fontsize{11pt}{12pt}\selectfont

\usepackage{geometry}	% 设置页边距的宏包
\usepackage{titlesec}   % 设置页眉页脚的宏包

\usepackage{fancyhdr}

\geometry{a4paper,scale=0.8}
%\geometry{a4paper,left=2cm,right=2cm,top=1cm,bottom=1cm}

\newpagestyle{main}{            
    \sethead{左页眉}{中页眉}{右页眉} %设置页眉
    \setfoot{左页脚}{中页脚}{右页脚} %设置页脚,可以在页脚添加 \thepage  显示页数
    \headrule   % 添加页眉的下划线
    \footrule   %添加页脚的下划线
}
%\pagestyle{main}    %使用该style
\fancyhf{}

\lhead{\textnormal{\kaishu\rightmark}}
\rhead{--\ \thepage\ --}
\pagestyle{fancy}

% \sectionmark 的重定义需要在 \pagestyle 之后生效
\renewcommand\sectionmark[1]{%
%\markright{\CTEXifname{\CTEXthesection——}{}#1}}
\markright{\CTEXifname{\CTEXthesection}{}#1}}

\usepackage{enumitem}
\setenumerate[1]{itemsep=0pt,partopsep=0pt,parsep=\parskip,topsep=0pt}
\setitemize[1]{itemsep=0pt,partopsep=0pt,parsep=\parskip,topsep=0pt}
\setdescription{itemsep=0pt,partopsep=0pt,parsep=\parskip,topsep=0pt}

%\lipsum[1]
%\clearpage

% pdflatex, xelatex

\title{3维点云数据适配性智能评估}
\title{设计工作报告}
\author{胡继承}
\date{\today}
\begin{document}

%\ctexset{section={name={第,节},number=\arabic{section}},}	% 第n节

%\ctexset{section={name=\S,number=\arabic{section}},}		% SS

%\ctexset{section={name=\S,number=\arabic{section}},
%subsection={number=\arabic{section}.\arabic{subsection}},}

%\ctexset{
%section={name={{\Huge\bf\S}},number={\normalsize{\arabic{section}}}},
%subsection={number=\arabic{section}.\arabic{subsection}},}

\ctexset{
chapter={name={第,章},number=\chinese{chapter},
nameformat={\heiti \zihao{1}\leftline },
titleformat={\heiti \zihao{1}\leftline }},
section={number={\arabic{chapter}.\arabic{section}},
nameformat={\heiti \zihao{2}},
titleformat={\heiti \zihao{2}}},}
%subsection={number=\arabic{section}.\arabic{subsection}},}

\ctexset{
section/format += \sffamily\raggedright,
%subsection/format += \fbox, % subsection 标题添加方框
}

%\ctexset{chapter={name={第,章},number=\chinese{chapter}},
%section={name={{\bf\S}},number={\normalsize{\arabic{section}}}},
%subsection={number=\arabic{section}.\arabic{subsection}},}

%-------------------------------  标题页  -------------------------------------
\begin{titlepage}
%\pagestyle{empty}
%\thispagestyle{empty}     %当前页码空,需紧跟在\maketitle后面使用
\pagenumbering{}	% 不显示页码
\songti\zihao{-4}	% 设置正文字体格式:宋体四号

% 首行缩进, 可以在全文之前设置首行缩进:\parindent=19pt
% 空格 \qquad \quad \  \; \, \! \!\!

名\quad 称:	3维点云数据适配性智能评估

单\quad 位:	武汉大学计算机学院

责任人:	胡继承

手\quad 机:	13871110050

\end{titlepage}

%-------------------------------------------------------------------

\maketitle

\frontmatter	% 前言部分,页码为小写罗马字母格式;其后的\chapter 不编号。


%-----------------------------------  中英摘要  -------------------------------------

%\chapter[中文摘要]{摘要}

%\chapter[英文摘要]{Abstract}


%------------------------------------   目录  ------------------------------------

\tableofcontents

\thispagestyle{empty}

%--------------------------------------  正文 --------------------------------------
\mainmatter		% 表示开始正文部分的内容 使用数字进行页面编号,对其中的chapter进行编号(第一章、第二章……)

%\newpage

\setcounter{page}{1}
\chapter{总体介绍}

国内外对3维点云数据适配性的各相关各领域已进行了大量的研究,但综合各种要素对3维点云数据适配性的设计
进行综合适配性评估分析还未见诸已发表的研究,而在工效学分析方面更是有着很大的欠缺。
现有的研究,更多侧重于传统的结构设计、气密性、材料配方、数字化设计等等。由于
未考虑基于大规模测量下不同物体点云数据存在的差异性特征,所以难以对点云数据间的适配性
展开研究。

{\heiti
本研究将基于大规模实测物体数据建立点云数据间的适配模型,结合人工智能算法建立适配评估系统。
}

\section{概述(任务来源及工程设计依据)}

本研究立足于3维点云数据适配性的设计和检测的行业应用需求,依照实测的大规模点云数据三维点
云数据,建立3维点云数据适配性的标准适配模型(数字模型);并能结合压力等其它传感参数,基于国
家标准 GB/T18664-2002《呼吸防护用品的选择、使用与维护》,采用人工智能算法来建立
3维点云数据适配性设计方案工效学适配性评估系统;在点云数据间相关典型行业具有现实的应用前景。

\section{产品的组成、功能及工作原理}

研究目标由3部分组成:

\subsection{标准适配模型}

采用分栏算法根据实测的大规模点云数据三维点云数据,计算出的标准适配数字模型。

标准适配模型的工作原理见附件。

\subsection{GB/T18664-2002 防护等级分析}

根据 GB/T18664-2002 标准在 IDLH 环境下对指定防护因素(APF, Assigned Protection 
Factor)的数据与3维点云数据适配性关联要素进行分析,得到相应的策略模型。

\subsection{点云数据间适配性评估系统}

结合前面2个部分并综合其它实测参数、采用深度学习的方法研发3维点云数据适配性的适配性评估系统。

\chapter{进展情况}

\section{主要战术技术指标和使用要求的实现情况}

\begin{itemize}
%\setlength{\itemsep}{0pt}
%\setlength{\parsep}{0pt}
\setlength{\parskip}{0pt}
\item[1)]
标准适配模型来自于对点云数据三维扫描数据进行科学分析计算得到,结果是真实可靠的
点云模型;
\item[2)]
建立的点云数据间适配模型符合 GB/T18664-2002《呼吸防护用品的选择、使用与维护》标准;
\item[3)]
建立的点云数据间工效学适配性评估系统适用于的物体点云数据特征。
\end{itemize}



\section{总体技术方案的实现情况}

已经完成第一、第二个研究目标,第3个研究目标正处于联调及总调阶段,目前进展顺利,大约
需要一个月时间完成调试。


\section{对研制过程中发生问题的分析处理情况}

在研发的过程中,我们发现如下两个问题:

\begin{itemize}
%\setlength{\itemsep}{0pt}
%\setlength{\parsep}{0pt}
\setlength{\parskip}{0pt}
\item[1)]
分区椭球面投影比柱面投影更能反映点云数据型态特征,需要改变投影方式;

\item[2)]
特征卷积网络不足以完整反映分区形态特征适配性评价,需要进一步研究更合理的人工神经网络。

\end{itemize}


\subsection{投影方式的改变}

初期研究计划将3维点云数据进行柱面投影,其后形成广义图片再进行后续处理。但我们发现
仅仅使用第一个研究目标的标准数字模型不是最恰当的工效学适配性评估的方式:3维点云数据适配性的
工效学适配性不仅仅针对面长面宽指标(可以用柱面投影完整描述),还应考虑纵深,也就是
说加上深度信息才能更好描述头部形态。考虑到点云数据形态与椭球体更相似,椭球投影比柱面
投影更能反映点云数据的型态特征。

再考虑到这些型态特征只有在统计意义下才能对优化3维点云数据适配性的设计有指导意义,而分区投影
不仅能提供整体特征,还能提供局部形态统计指标,从而对设计的指导意义更大。因此我们将
柱面投影调整为分区椭球面投影,并将孤立的按标准数字模型的投影改变为大数据下机器学习
获得的具有统计意义的量化投影。

上述改变大大增加了编程工作量和难度,但由于实际价值更高我们最终选择了这种更合理的
方法。

\subsection{人工神经网络结构的优化}

特征卷积网络优点是能提取广义图片的关键特征,但还不足以完整反映分区形态特征适配性
评价,需要进一步研究更合理的人工神经网络。经过数个月的研究我们发现特征卷积网络联合
连续态连续动作空间的深度强化学习网络能比较好地评估分区形态特征从而综合评测全局
形态适配性,给出更加细致的评估结果从而对被评估的设计方案的改进提供具有侧重点的更
细致的结论。因此我们优化了初始设计中的人工神经网络的结构方案,采用了更为复杂的
卷积网络接续强化学习网络的结构,以便真实反映形态特征的规律。

\section{第二个研究目标相关研究}

使用3维点云数据适配性的目的是预防有害环境威胁任务作业人员的健康,所以防护对象必须与危害存在
的形态相匹配,防护水平必须与危害程度相当,能够将危险水平降到可以接受的安全程度,还
必须方便任务作业,适合在实际作战条件下使用。与同时使用的作战工具或防护用品配合。并
适合使用人的特点。所以应先根据有害环境选择(对应于标准的4.2),再根据作业状况选择(对
应于标准的4.3),最后根据作业人员选择(对应于标准的4.4)来进行。

识别有害环境性质,判定危害程度,是选择3维点云数据适配性的第一步。


GB/T18664-2002《呼吸防护用品的选择、使用与维护》标准将有害环境的危害程度分成
两类:立即威胁生命和健康(IDLH, Immediately Dangerous to Life or Health)
环境和非 IDLH 环境。IDLH 环境是指呼吸危害能够使在其中没有得到呼吸防护的作业人
员致死货或丧失逃生能力或致残。包括 3 种情况:

\begin{itemize}
%\setlength{\itemsep}{0pt}
%\setlength{\parsep}{0pt}
\setlength{\parskip}{0pt}
\item[1)]
{\heiti
危害未知的环境:
}
当有理由怀疑需要进入的作业区中存在可能威胁生命的危险,但既没有控制措施将危险排除,
也没有检测手段加以判断,这就是危害未知的环境。

\item[2)]
{\heiti
缺氧未知和缺氧环境:
}
GB8958 规定空气中氧气浓度低于 18\% 的环境为缺氧环境。
\item[3)]
{\heiti
有害物浓度达到 IDLH 浓度的环境:
}
GB/T18664-2002 规范性附录B提供了 317 种空气污染物的IDLH浓度,是由美国国家职业安全
卫生研究所(NIOSH)研制的。由于 NIOSH 的 IDLH 浓度多数以 ppm 为单位,不符合我国的
有关规定,标准同时提供了将 1 ppm 换算成 mg/m3 的换算系数及换算结果,以方便使用。

\end{itemize}

除上述 3 种 IDLH 环境外,其它的呼吸危害环境就是非 IDLH 环境。在危害程度确
定后,就可选择防护等级与危害程度相匹配的呼吸防护用品。

\subsection{指定防护因子}

指定防护因子(APF, Assigned Protection Factor)是指一种或一类适宜功能的3维点云数据适配性,
在适合使用者佩戴且正确使用的前提下,预期能将污染物浓度降低的倍数。APF 越高表明其
安全性和可靠性越高,防护等级越高。

标定防护因子(NPF, Norminated Protection Factor)来自试验检测,在实验舱中使用
浓度恒定的气溶胶由佩戴点云数据间的检测模型实际测定出的舱内浓度和3维点云数据适配性内浓度的
比例(TIL, Total Inward Leakage percentage),是滤料、气阀和面罩密封的泄露
总和。

现场防护因子(WPF, Workplace Protection Factor)则是在实际作训过程中测量的,
是作训人员佩戴3维点云数据适配性从事实际作训过程中现场环境中污染物浓度与泄露进入3维点云数据适配性
内的浓度的比值,代表3维点云数据适配性的实际防护水平。

呼吸防护用品的指定防护因数APF是以
WPF为基础.由政府和标准化机构按照一定
方法“人为”制订的.代表各类呼吸防护用
品的某种可以接受的通常的防护能力。

我国缺少对呼吸防护用品现场应用的
研究,没有WPF的资料。本标准采取了比
较灵活的办法.参考国内部分呼吸防护用品产
品标准的TIL参数,本着保持与国外同类标准
基本一致的原则,制订了我国的APF。这样做虽有一定风险,但它是能够限制呼吸防护用品
应用范围的唯一方法,是本质性的进步,同时
提出现场研究的需求.以便逐步完善我国的
APF体系。


APF主要与3维点云数据适配性种类和3维点云数据适配性内空气压力的
高低两个因素有关.如果在吸气和呼气阶段3维点云数据适配性内压力始终高于大气压.就是正压式.否则
就是负压式。按固定压力因素进行分析,相对
于全面罩,半面罩较难取得与人脸口鼻区域
的密合.较易出现泄漏,所以同样是正压式或
是负压式.半面罩的APT总比全面罩低按固
定面罩种类进行分析.相对负压式.正压式面
罩内维持的正压可以防止外部有害物进入面
罩,安全防护性较高.APF也就比负压式的高,
如正压供气式半面罩的APF为50.自吸过滤式
和负压供气式半面罩的AFF都是10。


\section{系统与各分系统、各分系统之间、分系统与设备之间的接口协调性}

\section{标准化、质量、可靠性、维修性、测试性、保障性、安全性、环境适应性、电磁兼容性、
人机工程、测试、计量等质量特性的分析和实现程度}

\section{关键特性和重要特性分析,关键件、重要件的确定}

\section{通用化、系列化、组合化设计情况}

\section{设计验证试验情况及结果}

\section{元器件、原材料、零部件控制情况及结果}

\chapter{下一步工作}

\section{制定的定型样机试验方案}

\section{研制进度计划}

进入总调联调阶段计划:\\
1) 将前期研究工作的理论部分形成2篇论文投稿出去

2)点云数据点云数据的自动修复能大大增强系统的自动化性能,
目前在三维点云数据的反演上还存在一些问题(前期出现的
法向量与颜色格式冲突问题已经解决)。备选措施:如果短期
内出现障碍先以手工修复的模型作为标准数字模型进行系统调试,
再逐步解决。

3)在自设数据集上进行流程调试,完成后再增加数据集。
我们刚刚购买了rtx3080的GPU,比原先设想采用rtx2080先进很多。

4)检验人工神经网络结构设计的有效性,对特征卷积网络接续连续态连续动作空间
的输出效果进行验证(首先进行几何密合性验证便于评估效果)。

5)硬件上考虑增加点云数据姿态检测传感器(检测速度与转动姿态),将来好进行点云数据姿态与3维点云数据适配性压力分布的相关性检测与评估。(这个不是重点)

其中最后一项是为了投稿国内核心期刊准备增加的内容,如果可以投国外的话这一部分不是重点。


{\kaishu 本项目提出了一种新的3维点云数据适配性适配性评估系统,辅助压力数据的采集和分析。
该系统基于一个带嵌入式点阵传感器的可穿戴节点网络,采集的数据通过单片机上传到
处理控制中心进行分析处理。后续章节将讨论数据采集硬件的初始设计和数据分析算法。
实际的环境测试未进行,计划进行多种科目的测试,每个科目测试五次。}

{\songti 实验结果表明,该系统在获得3维点云数据适配性对点云数据的压力、压力的变化及动态参数
的影响等诸多方面具有良好的性能,为3维点云数据适配性适配性评估提供重要线索。}

{\fangsong 压力数据的动态变化(各种不同的姿势、运动状态)是一种复杂的过程,
它可能从侧面反映了3维点云数据适配性适配性能。不同的工作环境和作业方式可能导致应力不同的
分布形态,因此定量应力动态分析可以为3维点云数据适配性的设计提供重要的适配性评估参数。}

{\songti 
目前,3维点云数据适配性应力动态分析主要在专业的实验室进行,使用高精度光学系统和安装在
3维点云数据适配性内的精密应力点阵传感器来获取动态数据。实验室的分析可以提供精确和准确
的结果,但设备制造成本高昂且复杂。本项目通过视觉观察系统获取环境参数,通过
精密应力点阵传感器来获取应力数据,辅助设备相对较少,而结果定性且能对应力数据
与环境数据进行动态分析从而对3维点云数据适配性的性能进行精确地客观性评估。

作为3维点云数据适配性适配性评估系统,需要研发应力分布的测量装置,该装置要易于操作,能够
提供3维点云数据适配性对于佩戴者的动态的应力值进行准确动态测量和分析,其无需建立特殊实验
的环境。基于压电技术的薄膜应力测量点阵篇是一种低成本选择。由加速度计和陀螺仪组
成的广泛使用的惯性测量单元则可用于感知使用者的操作状态,将其用于点云数据动态分析。
使用可穿戴惯性测量单元的好处是体积小、成本低,测试不限于特定环境,并且可轻松与
压力传感器阵列进行集成,加上时间戳后方便进行相关性分析。

惯性测量单元还用于对复杂的头部动作的姿态数据进行采样。为了获得有用的头部姿态动作
参数,我们使用 C++ 在 Windows10 平台开发了基于数据融合算法的软件。定量头部姿态
参数,如左右转动速率、低头抬头速率,可以通过软件分析获得。获取的采样数据被上传并
保存在计算机中供后续进一步分析,亦可显示在软件的界面上进行直观观察分析。

本章的其余部分按如下方式组织:在第一节系统设计中,我们描述了整个系统的硬件和头部
姿态分析软件的架构设计。随后在第二节中给出了实验的设计及数据采集方法。最后我们在
第三节中进行总结,并规划未来的研究目标。
}

\section{系统设计}


\subsection{总体设计}

{\songti 
系统(图 1)基于带嵌入式传感器的有线节点网络。有线传感器的选择是,我们可以使用一个
锂聚合物电池来对整个采集单元供电。此外,使用有线可以有效地减少数据采集节点的重量
并减少数据传输损失。该系统主要由压力传感器阵列、多个MEMS惯性传感器单元组成的数据
采集节点、嵌入式控制板和上位计算机及数据分析处理软件组成。利用 ADIS16405 采集准确、
实时加速度和角速度数据的数据采集节点,通过 RS485 与控制板连接,数据传输速率为
115200bps。控制电路板主要负责:a)数据暂存; b)通过CAN网将数据传输到上位机; c)
时间戳管理。上位机中的软件接受头部姿态数据分析、压力传感器阵列数据,并可提供头部姿态
特征分析,如左右转动速率、低头抬头速率、专注静态阶段等。
}

\subsection{数据采集传感器单元}

{\songti 
利用4个惯性测量单元采集头部姿态动作的加速度和角速度数据,用于头部姿态的动态分析,
分别安装在头后部、下颌部和左耳及右耳处。数据采集节点的架构如图 2 所示。以达到尽
可能大的采样率。该单元采用以 Cortex-M3 内核的 32 位嵌入式 ARM 控制器,利用标准
接口(SPI、I2C、UART等)连接各种外设。惯性测量单元采用 ADIS16405 来对加速度和
角速度数据进行采样,W25Q128 闪存用于暂存采样数据,而后可以通过 SPI 访问。数据采
集控制单元使用 RS485 通信模块与数据采集传感单元通信,发送指令并和接受采集到的数据。
}

\subsection{数据采集控制单元}

{\songti 
}

\subsection{数据预处理 - 3D 点云数据自动修复}

\subsection{特征卷积网络与适配性匹配}

\section{关键算法}


\chapter{点云数据自动修复}

%\chapter[点云数据姿态与应力数据采集与分析]{\texorpdfstring{\\点云数据姿态与应力数据\\采集与分析}{点云数据姿态与应力数据采集与分析}}
\chapter{点云数据姿态与应力数据采集与分析}

\chapter{适配性评估分析系统的实现}



%--------------------------------------  总结和附录  --------------------------------------
\backmatter %后记部分,页码格式不变,继续正常计数;其后的\chapter 不编号。

%\chapter{总结与体会}
%\chapter{谢辞}
%\chapter{参考文献}

%----- 附录不编号但书签生效--------------
\chapter{附录}

\setcounter{secnumdepth}{0}   %设置section不带编号但是同时书签生效
\section{附录1 分栏算法及标准数字模型}
\section{附录2 }
\section{附录3 }
\section{附录4 }

\end{document}